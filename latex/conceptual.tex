\subsection{Descripción general}
Nuestro modelo conceptual tiene como principal clase el Proyecto, que es el eje de todo el Sistema y el agente con mas relaciones con los demás. A su vez, se modelan los presupuestos y contratos con los proveedores y clientes. Determinamos que cada presupuesto de un Proveedor es para un proyecto diferente pero que un Cliente puede tener varios presupuestos propuestos para un mismo proyecto. En caso de que esos presupuestos sean aprobados, se crea un Contrato para ese presupuesto (cosa que marcamos con una relación 0..1 contratoCliente/contratoProveedor).

Cada proyecto en sí tiene un Gerente que lo supervisa y un PM actual (pueden haber varios PMs relacionados mediante la clase Comision, pero solo uno es el actual, los demás están porque fueron PMs y en algun momento fueron cancelados). Este PM puede subir actualizaciones de Proyecto y por cada una se genera una actualización.

También modelamos los rankings de Proveedores y PMs, evaluaciones al terminar el proyecto y notificaciones según el agente que la recibe (es decir, si hay una notificación para, por ejemplo, un Gerente y un PM, se generan dos notificaciones distintas en este modelo, una para el Gerente y otra para el PM).
El caso de los Contadores decidimos modelarlo relacionándolo al Gerente y no al Proyecto porque el Contador ayuda el Gerente y las cobranzas y pagos lo realiza todo externamente al Sistema.

El caso en que un nuevo PM es asignado para un proyecto, se marca la Comisión del viejo PM como cancelada (mediante una propiedad de estado) y se agrega una Comisión a otro PM. En cambio, cuando hay un cambio de Proveedor, el Contrato de ese Proveedor es el que se marca como cancelado (también mediante una propiedad de estado).

Consideramos que tanto los contratos de los Proveedores pueden ser cancelados como los de los Clientes. En el primer caso porque un Proveedor puede ser removido de la obra y cambiado por otros. En el segundo porque al renegociar con nuevos Proveedores (cuando estos cambian), también hay que firmar un nuevo contrato con el cliente.

Notar que los Data Entries no los modelamos pues no eran relevantes para el Proyecto en sí.

\subsection{Relación con otros modelos}
Ahora bien, como explicabamos antes, el modelo conceptual modela los estados, propiedades y relaciones estáticas entre los agentes de un proyecto según el Sistema. Por esto mismo, hay cosas que los demás modelos especifican de forma mas precisa:

Las etapas previas a la creación de un proyecto, el acuerdo de alcance y duración del proyecto con el PM se ven mucho mejor en los diagramas de FSM y Actividad, de hecho, el alcance lo modelamos asumiendo que guardar todos los alcances acordados y re-acordados del Proyecto para el Modelo Conceptual es irrelevante y solo nos interesa el alcance actual, es decir el último acordado.
La autenticación y los procedimientos de un Usuario en el Sistema y cómo se usa el Sistema en si, está explicado más claramente en el Diagrama de Casos de Uso.

Las notificaciones, como esto afecta al sistema y qué eventos las causan se pueden ver explícitamente en los Diagramas de FSM. 

Los Data Entries también son modelados en los demás Diagramas. 

La cronología en general de todos los eventos del proyecto se explica mejor en el Diagrama de Actividad.

Los pagos y cobranzas del Contador a cada parte (Cliente, PM, Gerente, Proveedores) se especifica en el Diagrama de Actividad.

\subsection{Observaciones de las Vistas}
Como se puede ver, hay tres vistas para el Diagrama de Clases, pero estas tres vistas forman en realidad parte de un solo diagrama. Las separamos por motivos de claridad pero tienen que ser leídas e interpretadas como parte de un mismo Diagrama de Clases. Por este motivo, marcamos con números las clases que se repiten en las vistas para decir que en realidad son las mismas clases y que todo es parte de un solo y mismo diagrama.

\newpage
\subsection{Vistas}
\begin{center}
\includegraphics[scale=0.5, angle=90]{imagenes/ModeloConceptual1.png}
\end{center}
\newpage
\begin{center}
\includegraphics[scale=0.5, angle=90]{imagenes/ModeloConceptual2.png}
\end{center}
\newpage
\begin{center}
\includegraphics[scale=0.5, angle=90]{imagenes/ModeloConceptual3.png}
\end{center}

\newpage
\subsection{OCL}
Nuestro modelo tendrá las siguientes restricciones:

\begin{listocl}

\begin{itemocl}{Los rankings de Proveedor están bien formados (tienen del número 1 a cantidad(Proveedores)).}
Context: Proveedor
Inv: Proveedor.allInstances()->isUnique(posicionProv.numero) and Proveedor.allInstances()->exists(p | p.posicionProv.numero = 1) and Proveedor.allInstances()->exists(p | p.posicionProv.numero = Proveedor.allInstances()->size())
\end{itemocl}

\begin{itemocl}{Los rankings de PMs están bien formados (tienen del número 1 a cantidad(PMs)).}
Context: PM
Inv: PM.allInstances()->isUnique(posicionPM.numero) and PM.allInstances()->exists(p | p.posicionPM.numero = 1) and PM.allInstances()->exists(p | p.posicionPM.numero = PM.allInstances()->size())
\end{itemocl}

\begin{itemocl}{Cada proveedor tiene un seguro de caución que vence después de la finalización de cada proyecto al que le mandó un presupuesto.}
Context: Proveedor
Inv: self.presupuestos->collect(presupuestoProvDe)->forAll(proy | self.seguros->exists(seg | seg.fechaVencimiento >= proy.fechaFin))
\end{itemocl}

\begin{itemocl}{Los contratos de Proveedores están asociados a un gerente que supervisa el proyecto del contrato.}
Context: ContratoProveedor
Inv: self.gerenteInvolucrado.supervisa->includes(self.presupuesto.presupuestoProvDe)
\end{itemocl}

\begin{itemocl}{Los contratos de Clientes están asociados a un gerente que supervisa el proyecto del contrato.}
Context: ContratoCliente
Inv: self.gerenteInvolucrado.supervisa->includes(self.presupuesto.presupuestoClienteDe)
\end{itemocl}

\begin{itemocl}{Cada proveedor manda como máximo un presupuesto por proyecto.}
Context: Proveedor
Inv: self.presupuestos->isUnique(presupuestoProvDe)
\end{itemocl}

\begin{itemocl}{Hay como máximo un presupuesto del cliente con contrato firmado (y no cancelado) para cada proyecto.}
Context: Proyecto
Inv: self.presupuestosCliente->select(pres | pres.contrato->notEmpty() and pres.contrato.estado = firmado)->size() <= 1
\end{itemocl}

\begin{itemocl}{Si el proyecto no empezo, no hay contrato de clientes firmados, si empezo hay exactamente un contrato de cliente firmado y al menos un contrato de Proveedor firmado.}
Context: Proyecto
Inv: self.estado = noEmpezo implies self.presupuestosCliente->select(pres | pres.contrato->notEmpty() and pres.contrato.estado = firmado)->size() = 0 and self.estado <> noEmpezo implies (self.presupuestosCliente->select(pres | pres.contrato->notEmpty() and pres.contrato.estado = firmado)->size() = 1 and self.presupuestosProv->select(pres | pres.contrato->notEmpty() and pres.contrato.estado = firmado)->size() > 0)
\end{itemocl}

\begin{itemocl}{La suma de los presupuestos de los Proveedores que firmaron un contrato es menor al presupuesto que se le manda al cliente.}
Context: Proyecto
Inv: self.presupuestosCliente->select(presCliente | presCliente.contrato->notEmpty() and presCliente.contrato.estado = firmado)->forall(presCliente | presCliente.monto < self.presupuestosProv->select(presProv | presProv.contrato->notEmpty() and presProv.contrato.estado = firmado)->collect(monto)->sum())
\end{itemocl}

\begin{itemocl}{El cliente de un proyecto coincide con el cliente asociado a cada presupuesto del proyecto.}
Context: Proyecto
Inv: self.presupuestosCliente->forAll(pres | pres.cliente = self.cliente)
\end{itemocl}

\begin{itemocl}{Hay como máximo un PM actual por proyecto (puede ser momentaneamente ninguno cuando se cambia de PM).}
Context: Proyecto
Inv: self.comisionesProyecto->select(c | c.estado = actual) <= 1
\end{itemocl}

\begin{itemocl}{Cada comisión de un PM es para un proyecto distinto.}
Context: PM
Inv: self.comisionesPM->isUnique(proyecto)
\end{itemocl}

\begin{itemocl}{Todas las actualizaciones de Proyecto pertenecen a un PM que es o fue el PM asignado del proyecto de la actualización.}
Context: PM
Inv: self.actualizaciones->forAll(a | self.comisionesPM->exists(c | c.proyecto = a.actualizacionDe))
\end{itemocl}

% ----------- Evaluaciones

\begin{itemocl}{Las evaluaciones de Gerentes a PMs están en un ciclo válido (es decir, el PM y el Gerente estuvieron asignados al proyecto de la evaluación).}
Context: EvaluacionGerentePM
Inv: self.PM.comisionesPM->collect(proyecto)->includes(self.proyecto) and self.gerente.supervisa->includes(self.proyecto)
\end{itemocl}

\begin{itemocl}{Las evaluaciones de Clientes a PMs están en un ciclo válido (es decir, el Cliente y el Gerente estuvieron asignados al proyecto de la evaluación).}
Context: EvaluacionClientePM
Inv: self.PM.comisionesPM->collect(proyecto)->includes(self.proyecto) and self.cliente = self.proyecto.cliente
\end{itemocl}

\begin{itemocl}{Las evaluaciones de Gerentes a PMs están en un ciclo válido (es decir, el PM y el Gerente estuvieron asignados al proyecto de la evaluación).}
Context: EvaluacionPMProveedor
Inv: self.PM.comisionesPM->collect(proyecto)->includes(self.proyecto) and self.proyecto.presupuestosProv->select(p | p.contrato->notEmpty())->includes(self.proveedor)
\end{itemocl}

\begin{itemocl}{Por cada PM cancelado hay una evaluación del Cliente al PM y otra del Gerente al PM.}
Context: Comision
Inv: self.estado = cancelada implies (self.proyecto.evalsGerPMProyecto->exists(e | e.PM = self.comisionDe) and self.proyecto.evalsClientePMProyecto->exists(e | e.PM = self.comisionDe))
\end{itemocl}

\begin{itemocl}{Por cada Proveedor cancelado hay una evaluación del PM a ese Proveedor.}
Context: ContratoProveedor
Inv: self.estado = cancelado implies self.presupuesto.presupuestoProvDe.evalsPMProvProyecto->exists(e | e.proveedor = self.presupuesto.proveedor)
\end{itemocl}

\begin{itemocl}{Si el proyecto terminó, se evaluaron a todos los PMs y Proveedores involucrados.}
Context: Proyecto
Inv: self.estado = terminado implies (self.comisionesProyecto->collect(comisionDe)->forAll(pm | self.evalsGerPMProyecto->includes(e | e.PM = pm) and self.evalsClientePMProyecto->exists(e | e.PM = pm)) and self.presupuestosProv->select(pres | pres.contrato->notEmpty())->collect(proveedor)->forAll(prov | self.evalsPMProvProyecto->exists(e | e.proveedor = prov)))
\end{itemocl}

% ----------- Notificaciones

\end{listocl}