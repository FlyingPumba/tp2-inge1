Proponemos como alternativas los siguientes opciones:
\begin{itemize}
	\item Generación e utilización de encuestas sobre el desempeño de los PM y de los proveedores involucrados en los proyectos.
	\item Incluir actualizaciones sobre el estado de los proyectos en el sistema, periódicas (donde el contenido de este tipo de actualización no se refiere a problemas graves en los proyectos, tales como cancelación de proveedores o atrasos en los mismos) y sobre problemas graves o sólo incluir actualizaciones sobre problemas graves en el sistema.
\end{itemize}

Con respecto a la generación e utilización de encuestas, esta alternativa tiene diversos efectos sobre el accionar de los agentes de la empresa. Por un lado genera una carga laboral adicional para los gerentes y los PM, ya que deben completar una encuesta sobre el desempeño del PM en el proyecto y sobre los proveedores respectivamente. También requiere una evaluación de los clientes sobre los PM, lo cual puede generar un pequeño malestar en los clientes. Sin embargo, esta pequeña carga adicional, redunda en mejoras a la hora de seleccionar proveedores y PM, debido a que permite la generación de rankings usando como input las encuestas de proyectos previos. De esta manera se agiliza la selección de proveedores y la selección de PM.

Por otro lado, la alternativa incluir actualizaciones periódicas y sobre problemas graves de los proyectos en el sistema tiene un efecto positivo sobre el conocimiento de los PM y los gerentes sobre el estado de cada proyecto. Este efecto positivo surge del hecho que, en cualquier momento, tanto los gerentes como los PM conocen en que situación se encuentra cada proyecto, pudiendo anticiparse a problemas, y conocen sobre situaciones que requieren su inmediata intervención.
Sin embargo esta alternativa requiere una carga laboral mayor para los PM, debido a que estos estan obligados a actualizar el estado de cada proyecto a su cargo de forma periódica, no sólo cuando hay problemas.

Por ultimo nos parece importante mencionar que, en caso de que la proporción de proveedores que utilizan el sistema no sea alta, el sistema no va a proveer la escalabilidad deseada. Este defecto se debe a que, dado la gran cantidad de proveedores que maneja DC Construcciones, los pedidos de actualización de los seguros de caución representarían una carga laboral importante sobre los data entry, sumado al hecho que cada vez que un PM necesite presupuestos, este se deberá conectar telefónicamente con cada uno y luego cargar su propuesta, hecho que suma una cargar laboral importante sobre el PM.
