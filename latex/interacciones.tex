Para lograr comunicar el comportamiento del sistema a desarrollar utilizamos diversas herramientas (casos de uso, máquinas de estados finitos, diagrama de actividad y modelo conceptual), debido a que cada una de ellas tiene distinto poder expresivo.
Dado esta caracteristica de las herramientas de especificación surgen escenarios que son transversales a los distintos modelos generados, por lo cual es necesario aclarar como interactuan y se complementan, las herramientas, entre si.

Un paso previo al objetivo antes mencionado es detallar como y para que utilizamos cada herramienta en particular.
\subsection{Casos de uso}
Como mencionamos antes, el modelo de casos de uso nos permite representar las interacciones que existen entre
el sistema y los actores externos a este. Utilizando esta técnica podemos definir el alcance del propio sistema
en cuanto a que acciones toma ante cada interacción y, además, nos permite definir la interfaz que nuestro
sistema trendrá.

Apesar de todo lo dicho, este modelo es incapaz de modelar todas las interacciones que ocurran fuera del sistema.
Esta no es la única limitación que posee, dado que si bien podemos modelar un cierto orden en cuanto a los procesos
que la empresa realiza en el sistema, utilizando las Pre y Post condiciones de los detalles, no podemos dejar
claro ningun paralelismo que pueda existir entre ellas. Para eso y para otras cosas serán necesarios los demás modelos.

\subsection{Diagrama de Actividad}
El primer diagrama de actividad especifica un escenario completo del ciclo de vida de un proyecto con la particularidad de que suponemos que tanto el cliente como los proveedores utilizan el sistema, que, tanto el alcance como el presupuesto, son aceptados en el primer intento y que no hay inconvenientes en la etapa de seguimiento de los proyectos, es decir: el proveedor no cancela, el PM no es reemplazado y no cancela, y el PM carga las actualizaciones periodicas. Aclaramos que en este escenario el PM pide presupuestos a solo dos proveedores (llamados proveedor 1 y 2). Tomamos esta decisión porque el comportamiento del sistema no difiere si los pedidos de presupuesto son a 2 o a $N$ proveedores, mientras que con 2 proveedores ganamos claridad. Con respecto al resto de las caracteristicas particulares de este escenario, consideramos que para lograr captar al maximo el comportamiento del sistema y a su vez lograr que el diagrama sea legible, asumir que el cliente y el proveedor utilizan siempre el sistema es logico, lo mismo sucede con la definición del alcance y de presupuesto (proceso que se encuentra especificado en su totalidad en otra herramienta). Con respecto al seguimiento, consideramos que especificar todas las situaciones posibles en esta etapa en un diagrama resulta imposible por lo cual especificamos la situación ideal. El resto de los situaciones serán especificados en otros diagramas.   

El segundo diagrama de actividad especifica un escenario en donde el proveedor contratado (llamado proveedor 1 en el driagrama) cancela y se debe resolver esta situación extraordinaria. El mismo finaliza cuando un nuevo proveedor es contratado para llevar adelante la obra. Este escenario representa una de las situaciones no especificadas en el primer diagrama, y, dado su importancia, consideramos necesario mostrarlo con un diagrama propio.

El tercer diagrama de actividad especifica un escenario en donde el PM asignado es reemplazado debido a la inconformidad del cliente con respecto a su trabajo. El mismo finaliza con el reemplazo del PM por otro. Este escenario representa una de las situaciones no especificadas en el primer diagrama, y, dado su importancia, consideramos necesario mostrarlo con un diagrama propio.

Por ultimo el cuarto diagrama de actividad especifica un escenario en donde el PM asignado es cancela por decision propia. El mismo finaliza con el cambio del PM. Este escenario representa una de las situaciones no especificadas en el primer diagrama, y, dado su importancia, consideramos necesario mostrarlo con un diagrama propio.

\subsection{Máquinas de estados finitos}

\subsection{Casos de uso - Diagrama de Actividad}

\subsection{Diagrama de Actividad - Máquinas de estados finitos}
