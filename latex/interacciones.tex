Para lograr comunicar el comportamiento del sistema a desarrollar utilizamos diversas herramientas (casos de uso, máquinas de estados finitos, diagrama de actividad y modelo conceptual), debido a que cada una de ellas tiene distinto poder expresivo.
Dado esta caracteristica de las herramientas de especificación surgen escenarios que son transversales a los distintos modelos generados, por lo cual es necesario aclarar como interactuan y se complementan, las herramientas, entre si.

Un paso previo al objetivo antes mencionado es detallar como y para que utilizamos cada herramienta en particular (ESTO ESTARIA EN LA INTRO?)
\subsection{Casos de uso}

\subsection{Diagrama de Actividad}
El primer diagrama de actividad especifica un escenario completo del ciclo de vida de un proyecto con la particularidad de que suponemos que tanto el cliente como los proveedores utilizan el sistema, que, tanto el alcance como el presupuesto, son aceptados en el primer intento y que no hay inconvenientes en la etapa de seguimiento de los proyectos, es decir: el proveedor no cancela, el PM no es reemplazado y el PM carga las actualizaciones periodicas. Aclaramos que en este escenario el PM pide presupuestos a solo dos proveedores (llamados proveedor 1 y 2). Decidimos especificar este escenario porque consideramos que logra co
JUSTIFICAR PORQUE

El segundo diagrama de actividad especifica un escenario en donde el proveedor contratado (llamado proveedor 1 en el driagrama) cancela y se debe resolver esta situación extraordinario. El mismo finaliza cuando un nuevo proveedor es contratado para llevar adelante la obra. Decidimos especificar este escenario porque consideramos

Por ultimo el tercer diagrama de actividad especifica un escenario en donde el PM asignado es reemplazado debido a la inconformidad del cliente con respecto a su trabajo. El mismo finaliza con el reemplazo del PM por otro. Decidimos especificar este escenario porque consideramos

\subsection{Máquinas de estados finitos}

\subsection{Modelo Conceptual}

A partir de la descripción hecho, podemos establecer las siguientes relaciones:
\subsection{Casos de uso - Diagrama de Actividad}

\subsection{Diagrama de Actividad - Máquinas de estados finitos}

\subsection{Modelo Conceptual??}