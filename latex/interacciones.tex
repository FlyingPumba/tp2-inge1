Para lograr comunicar el comportamiento del sistema a desarrollar utilizamos diversas herramientas (casos de uso, máquinas de estados finitos, diagrama de actividad y modelo conceptual), debido a que cada una de ellas tiene distinto poder expresivo.
Dado esta caracteristica de las herramientas de especificación surgen escenarios que son transversales a los distintos modelos generados, por lo cual es necesario aclarar como interactuan y se complementan, las herramientas, entre si.

Un paso previo al objetivo antes mencionado es detallar como y para que utilizamos cada herramienta en particular (ESTO ESTARIA EN LA INTRO?)
\subsection{Casos de uso}
Como mencionamos antes, el modelo de casos de uso nos permite representar las interacciones que existen entre
el sistema y los actores externos a este. Utilizando esta técnica podemos definir el alcance del propio sistema
en cuanto a que acciones toma ante cada interacción y, además, nos permite definir la interfaz que nuestro
sistema trendrá.

Apesar de todo lo dicho, este modelo es incapaz de modelar todas las interacciones que ocurran fuera del sistema.
Esta no es la única limitación que posee, dado que si bien podemos modelar un cierto orden en cuanto a los procesos
que la empresa realiza en el sistema, utilizando las Pre y Post condiciones de los detalles, no podemos dejar
claro ningun paralelismo que pueda existir entre ellas. Para eso y para otras cosas serán necesarios los demás modelos.

\subsection{Diagrama de Actividad}
El primer diagrama de actividad especifica un escenario completo del ciclo de vida de un proyecto con la particularidad de que suponemos que tanto el cliente como los proveedores utilizan el sistema, que, tanto el alcance como el presupuesto, son aceptados en el primer intento y que no hay inconvenientes en la etapa de seguimiento de los proyectos, es decir: el proveedor no cancela, el PM no es reemplazado y el PM carga las actualizaciones periodicas. Aclaramos que en este escenario el PM pide presupuestos a solo dos proveedores (llamados proveedor 1 y 2). Decidimos especificar este escenario porque consideramos que logra co
JUSTIFICAR PORQUE

El segundo diagrama de actividad especifica un escenario en donde el proveedor contratado (llamado proveedor 1 en el driagrama) cancela y se debe resolver esta situación extraordinario. El mismo finaliza cuando un nuevo proveedor es contratado para llevar adelante la obra. Decidimos especificar este escenario porque consideramos

Por ultimo el tercer diagrama de actividad especifica un escenario en donde el PM asignado es reemplazado debido a la inconformidad del cliente con respecto a su trabajo. El mismo finaliza con el reemplazo del PM por otro. Decidimos especificar este escenario porque consideramos

\subsection{Máquinas de estados finitos}

\subsection{Modelo Conceptual}

Este modelo especifica de forma estática los diferentes estados válidos de un proyecto según el sistema. Con eso en mente, nuestro modelo realizado contempla:

\begin{itemize}
	\item Todo los agentes relacionados con el proyecto desde el punto de vista del Sistema
	\item Los presupuestos clientes y proveedores y como se relacionan con los contratos que luego se firman
	\item Comisiones de los PMs
	\item Los rankings de PMs y Proveedores
	\item Seguros de caución de los Proveedores y su validez
	\item Representa estados que muestran cuando hay cambios de PMs y Proovedores
	\item Actualizaciones de Proyecto
	\item Notificaciones periódicas y por problemas
	\item Evaluaciones de proyecto cuando el mismo finaliza
	\item Los estados y propiedades de un proyecto y todos sus agentes relacionados según el Sistema
\end{itemize}

\subsection{Casos de uso - Diagrama de Actividad}

\subsection{Diagrama de Actividad - Máquinas de estados finitos}


\subsection{Modelo Conceptual??}
