\subsection{Interacciones con  el sistema: representado con casos de uso}

En primera instancia tenemos que representar las acciones que los diversos actores realizan en el sistema o a traves de él. Para esto modelaremos nuestro proyecto usando casos de uso, y
mostraremos los detalles de cada uno de estos.

\begin{figure}[H]
    \centering
    %\includegraphics[width=\textwidth]{imagenes/casos-de-uso1.png}
\end{figure}

\begin{figure}[H]
    \centering
    %\includegraphics[width=\textwidth]{imagenes/casos-de-uso2.png}
\end{figure}

\begin{table}[htbp]
  \begin{center}
  \resizebox{\textwidth}{!}{
  \begin{tabular}{|l|l|}
  \hline
  \multicolumn{2}{|l|}{CU1: Registrar} \\\hline
  \multicolumn{2}{|l|}{Actor: Cliente} \\\hline
  \multicolumn{2}{|l|}{Pre: true} \\\hline
  \multicolumn{2}{|l|}{Post: El cliente está registrado en el sistema} \\\hline
   Curso normal & Curso alternativo\\ \hline
   1- El cliente solicita al sistema que le permita registrarse & \\ \hline
   2- El sistema confirma y pide al cliente que ingrese su nombre y clave & \\ \hline
   3- El cliente ingresa su nombre y clave & 3.1- El cliente ingresa un nombre o clave inválidos. Ir a Fin CU1\\ \hline
   4- El sistema guarda el nombre y la clave del cliente & \\ \hline
   5- Fin CU1 & \\ \hline
  \end{tabular}
}
%\caption{Tabla casos de uso}
%\label{tabla:casos de uso}
\end{center}
\end{table}

\begin{table}[htbp]
  \begin{center}
  \resizebox{\textwidth}{!}{
  \begin{tabular}{|l|l|}
    \hline
    \multicolumn{2}{|l|}{CU2: Enviar propuesta} \\\hline
    \multicolumn{2}{|l|}{Actor: Cliente} \\\hline
    \multicolumn{2}{|l|}{Pre: El cliente está registrado en el sistema} \\\hline
    \multicolumn{2}{|l|}{Post: El sistema guarda la propuesta del cliente, para que después la obtenga el gerente} \\\hline
     Curso normal & Curso alternativo\\ \hline
     1- El cliente solicita al sistema que le permita enviar una propuesta & \\ \hline
     2- El sistema confirma y pide al cliente que ingrese su nombre y clave & \\ \hline
     3- El cliente ingresa su nombre, clave y su propuesta  & 3.1- El cliente ingresa un nombre o cleve incorrectos. Ir a Fin CU2\\ \hline
     4- El sistema guarda la propuesta del cliente & \\ \hline
     5- Fin CU2 & \\ \hline
  \end{tabular}
}
%\caption{Tabla casos de uso}
%\label{tabla:casos de uso}
  \end{center}
\end{table}

\begin{table}[htbp]
  \begin{center}
  \resizebox{\textwidth}{!}{
  \begin{tabular}{|l|l|}
    \hline
    \multicolumn{2}{|l|}{CU3: Acepta o rechaza presupuesto} \\\hline
    \multicolumn{2}{|l|}{Actor: Cliente} \\\hline
    \multicolumn{2}{|l|}{Pre: El cliente acepto el alcance} \\\hline
    \multicolumn{2}{|l|}{Post: El sistema guarda la aceptación del cliente} \\\hline
     Curso normal & Curso alternativo\\ \hline
     1- El cliente recibe el presupuesto & \\ \hline
     2- El cliente acepta la propuesta & 2.1- El cliente rechaza el presupuesto. Ir a CU(gerente envía presupuesto)\\ \hline
     3- El sistema guarda la aceptación del cliente &\\ \hline
     4- Fin CU3 & \\ \hline
  \end{tabular}
}
%\caption{Tabla casos de uso}
%\label{tabla:casos de uso}
  \end{center}
\end{table}

\begin{table}[htbp]
  \begin{center}
  \resizebox{\textwidth}{!}{
  \begin{tabular}{|l|l|}
    \hline
    \multicolumn{2}{|l|}{CU4: Acepta o rechaza alcance} \\\hline
    \multicolumn{2}{|l|}{Actor: Cliente} \\\hline
    \multicolumn{2}{|l|}{Pre: El cliente envío una propuesta} \\\hline
    \multicolumn{2}{|l|}{Post: El sistema guarda la aceptación del cliente} \\\hline
     Curso normal & Curso alternativo\\ \hline
     1- El cliente recibe el alcance & \\ \hline
     2- El cliente acepta el alcance & 2.1- El cliente rechaza el presupuesto. Ir a CU(gerente envía alcance)\\ \hline
     3- El sistema guarda la aceptación del cliente  &\\ \hline
     4- Fin CU4 & \\ \hline
  \end{tabular}
}
%\caption{Tabla casos de uso}
%\label{tabla:casos de uso}
  \end{center}
\end{table}

\begin{table}[htbp]
  \begin{center}
  \resizebox{\textwidth}{!}{
  \begin{tabular}{|l|l|}
    \hline
    \multicolumn{2}{|l|}{CU5: Sistema pide evaluación de PM al finalizar el proyecto} \\\hline
    \multicolumn{2}{|l|}{Actor: Cliente} \\\hline
    \multicolumn{2}{|l|}{Pre: El proyecto finalizo} \\\hline
    \multicolumn{2}{|l|}{Post: El cliente recibe el pedido de evaluación de un PM} \\\hline
     Curso normal & Curso alternativo\\ \hline
     1- El sistema se actualiza y observa que tiene un proyecto como finalizado & \\ \hline
     2- El sistema guarda el pedido de evaluación del PM para que el cliente lo observe. &\\ \hline
     3- Fin CU5 & \\ \hline
  \end{tabular}
}
%\caption{Tabla casos de uso}
%\label{tabla:casos de uso}
  \end{center}
\end{table}

\begin{table}[htbp]
  \begin{center}
  \resizebox{\textwidth}{!}{
  \begin{tabular}{|l|l|}
    \hline
    \multicolumn{2}{|l|}{CU6: Evalúa PM} \\\hline
    \multicolumn{2}{|l|}{Actor: Cliente} \\\hline
    \multicolumn{2}{|l|}{Pre: El cliente recibio del sistema un pedido de evalución para el PM} \\\hline
    \multicolumn{2}{|l|}{Post: El cliente evalúa al PM} \\\hline
     Curso normal & Curso alternativo\\ \hline
     1- El cliente pide al sistema que le permita evaluar al PM & \\ \hline
     2- El sistema confirma y pide al cliente su nombre y clave. & \\ \hline
     3- El cliente ingresa su nombre y clave & 3.1- El cliente ingresa un nombre o clave incorrectos. Ir a FinCU6\\ \hline
     4- El sistema confirma al cliente y muestra la encuesta para evaluar al PM & \\ \hline
     5- El cliente completa la encuesta y pide al sistema que la guarde & \\ \hline
     6- El sistema guarda la encuesta & \\ \hline
     7- Fin CU6 & \\ \hline
  \end{tabular}
}
  %\caption{Tabla casos de uso}
  %\label{tabla:casos de uso}
  \end{center}
\end{table}

\begin{table}[htbp]
  \begin{center}
  \resizebox{\textwidth}{!}{
  \begin{tabular}{|l|l|}
    \hline
    \multicolumn{2}{|l|}{CU7: Registrarse} \\\hline
    \multicolumn{2}{|l|}{Actor: Proveedor} \\\hline
    \multicolumn{2}{|l|}{Pre: true} \\\hline
    \multicolumn{2}{|l|}{Post: El proveedor queda registrado en el sistema} \\\hline
     Curso normal & Curso alternativo\\ \hline
     1- El proveedor solicita al sistema que le permita registrarse & \\ \hline
     2- El sistema confirma y pide al proveedor que ingrese su nombre y clave & \\ \hline
     3- El proveedor ingresa su nombre y clave & 3.1- El proveedor ingresa un nombre o clave inválidos. Ir a Fin CU7\\ \hline
     4- El sistema guarda el nombre y la clave del proveedor & \\ \hline
     5- Fin CU7 & \\ \hline
  \end{tabular}
}
%\caption{Tabla casos de uso}
%\label{tabla:casos de uso}
  \end{center}
\end{table}

\begin{table}[htbp]
  \begin{center}
  \resizebox{\textwidth}{!}{
  \begin{tabular}{|l|l|}
    \hline
    \multicolumn{2}{|l|}{CU8: Actualizar sus datos} \\\hline
    \multicolumn{2}{|l|}{Actor: Proveedor} \\\hline
    \multicolumn{2}{|l|}{Pre: El proveedor está autenticado} \\\hline
    \multicolumn{2}{|l|}{Post: El proveedor tiene actualizada su información en el sistema} \\\hline
     Curso normal & Curso alternativo\\ \hline
     1- El proveedor solicita al sistema que le permita actualizar sus datos & \\ \hline
     2- El sistema confirma y muestra los datos actuales del proveedor & \\ \hline
     3- El proveedor modifica sus datos &\\ \hline
     4- El sistema guarda los nuevos datos del proveedor & \\ \hline
     5- Fin CU8 & \\ \hline
  \end{tabular}
}
%\caption{Tabla casos de uso}
%\label{tabla:casos de uso}
  \end{center}
\end{table}

\begin{table}[htbp]
  \begin{center}
  \resizebox{\textwidth}{!}{
  \begin{tabular}{|l|l|}
    \hline
    \multicolumn{2}{|l|}{CU9: Actualizar datos de seguro de caución} \\\hline
    \multicolumn{2}{|l|}{Actor: Proveedor} \\\hline
    \multicolumn{2}{|l|}{Pre: El proveedor está autenticado} \\\hline
    \multicolumn{2}{|l|}{Post: El proveedor tiene actualizado los datos del seguro de caución} \\\hline
     Curso normal & Curso alternativo\\ \hline
     1- El proveedor solicita al sistema que le permita actualizar sus datos de seguro de caución & \\ \hline
     2- El sistema confirma y muestra los datos actuales del proveedor & \\ \hline
     3- El proveedor carga su nuevo seguro de caución &\\ \hline
     4- El sistema guarda los nuevos datos del proveedor & \\ \hline
     5- Fin CU9 & \\ \hline
  \end{tabular}
}
%\caption{Tabla casos de uso}
%\label{tabla:casos de uso}
  \end{center}
\end{table}

\begin{table}[htbp]
  \begin{center}
  \resizebox{\textwidth}{!}{
  \begin{tabular}{|l|l|}
    \hline
    \multicolumn{2}{|l|}{CU10: Propone presupuesto} \\\hline
    \multicolumn{2}{|l|}{Actor: Proveedor} \\\hline
    \multicolumn{2}{|l|}{Pre: El proveedor está autenticado} \\\hline
    \multicolumn{2}{|l|}{Post: El proveedor tiene actualizado los datos del seguro de caución} \\\hline
     Curso normal & Curso alternativo\\ \hline
     1- El proveedor solicita al sistema que le permita actualizar sus datos de seguro de caución & \\ \hline
     2- El sistema confirma y muestra los datos actuales del proveedor & \\ \hline
     3- El proveedor carga su nuevo seguro de caución &\\ \hline
     4- El sistema guarda los nuevos datos del proveedor & \\ \hline
     5- Fin CU10 & \\ \hline
  \end{tabular}
}
%\caption{Tabla casos de uso}
%\label{tabla:casos de uso}
  \end{center}
\end{table}

\newpage
