En el anterior informe se especifico una lista de requerimientos y se establecieron objetivos, con los cuales,
se llevaría acabo el sistema pedido por la empresa \textit{DC Construcciones}. A partir de esto,
elegiremos cuales de las alternativas propuestas en el modelo de objetivos utilizaremos para la
contrucción de este sistema. Una vez completada la elección, se realizaran una serie de documentos
en base a distintas técnicas de especificación, entre las cuales contaran: modelo de casos de uso,
modelo conceptual ampliados con OCL, máquinas de estados finitos (o LTSs/FSMs), y/o diagramas
de actividad. Según consideremos conveniente elegiremos una o otra para represandar determinados
detalles de nuestro problema.

A continuación redactaremos las elecciones que tomamos a partir del modelo de objetivos:

\begin{itemize}
\item 

\end{itemize}

Seleccionadas las alternativas, pasaremos establecer con que modelos representaremos cada detalle
 de nuestro problema.

 \begin{itemize}
  \item Casos de uso: usaremos esta técnica para representar las interacciones que ocurren entre los distintos
  actores que conforman la empresa(gerente, PM, proveedor, cliente, data entry) y el sistema. Completaremos
  el diagrama con los detalles de los casos de uso, donde presentaremos la serie de pasos que representan una
  determinada interección con nuestro sistema.
 \end{itemize}
