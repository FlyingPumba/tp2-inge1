En el anterior informe se especifico una lista de requerimientos y se establecieron objetivos, con los cuales,
se llevaría acabo el sistema pedido por la empresa \textit{DC Construcciones}. A partir de esto,
elegiremos cuales de las alternativas propuestas en el modelo de objetivos utilizaremos para la
construcción de este sistema. Una vez completada la elección, se realizaran una serie de documentos
en base a distintas técnicas de especificación, entre las cuales contaran: modelo de casos de uso,
modelo conceptual ampliados con OCL, máquinas de estados finitos (o LTSs/FSMs), y/o diagramas
de actividad. Según consideremos conveniente elegiremos una o otra para representar determinados
detalles de nuestro problema.

A continuación redactaremos las elecciones que tomamos a partir del modelo de objetivos:

\begin{itemize}
\item En el o-refinamiento del objetivo Mantener[Mostrar proveedores de forma inteligente] seleccionamos la opción que contiene los objetivos Mantener[Proveedores ordenados por ranking] y Lograr[Aplicar filtros para proveedores (rubro, ubicación, geografía)]. El ranking mencionado es generado a partir de las encuestas de fin de proyecto.
\item En el o-refinamiento del objetivo Mantener[orden de PM según ranking y cantidad de proyectos actuales] seleccionamos la opción que ordena a los PMs por ranking y cantidad de proyectos. Este ranking es generado a partir de las encuestas de fin de proyecto.
\item En el o-refinamiento del objetivo Mantener[Estado de proyectos actualizados] elegimos la opción que incluye como subobjetivo Mantener[PM asignado al proyecto
actualiza el estado periódicamente en el sistema].
\item En el o-refinamiento del objetivo Lograr[Mejorar finalización de los proyectos] elegimos la opción que incluye como subobjetivo Lograr[El proyecto se marca como finalizado => entonces se realizan las encuestas]. Esta elección es necesaria para que los dos primeros o-refinamientos se puedan utilizar.
\end{itemize}

Seleccionadas las alternativas, pasaremos establecer con que modelos representaremos cada detalle de nuestro problema.

 \begin{itemize}
  \item Casos de uso: usaremos esta técnica para representar las interacciones que ocurren entre los distintos
  actores que conforman la empresa(gerente, PM, proveedor, cliente, data entry) y el sistema. Completaremos
  el diagrama con los detalles de los casos de uso, donde presentaremos la serie de pasos que representan una
  determinada interacción con nuestro sistema.
  \item Diagrama de Actividad: usaremos esta técnica para representar el ciclo de vida de un proyecto, y casos extraordinarios en la etapa de seguimiento, cancelación del proveedor y cambio de PM.
  \item Máquinas de estados finitos: las utilizamos para mostrar como funcionan las actualizaciones de estado que realiza el PM para un proyecto
  durante la etapa de seguimiento (con y sin data entry). Además, mostramos como son las interacciones a través del sistema durante las negociaciones
  del alcance (PM con Cliente) y del presupuesto (Gerente con Cliente).
  \item Modelo Conceptual: usaremos esta herramienta para modelar los diferentes agentes relevantes desde el punto de vista del sistema así como también sus diferentes estados y relaciones entre los mismos. Es decir, con el modelo conceptual hablaremos sobre la estructura misma de las entidades del problema y como el sistema entiende todo lo involucrado según las características que los hacen relevantes.
 \end{itemize}

En las siguientes secciones de este informe empezaremos por formular cambios respecto a la etapa anterior del proyecto, para luego desarrollar cada herramienta utilizada, partiendo de una descripción general de las mismas para luego aclarar las relación entre la herramienta analizada y el resto de las herramientas utilizadas. La necesidad de aclarar esta relación surge del hecho de que cada herramienta tiene distinto poder expresivo, por lo tanto
para lograr comunicar el comportamiento del sistema a desarrollar debemos utilizar cada una para diversas situaciones, lo cual genera escenarios que son transversales a los distintos modelos generados, es decir, para especificarlos, es necesario utilizar más de una herramienta.
