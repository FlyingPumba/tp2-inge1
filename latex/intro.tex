En el anterior informe se especifico una lista de requerimientos y se establecieron objetivos, con los cuales,
se llevaría acabo el sistema pedido por la empresa \textit{DC Construcciones}. A partir de esto,
elegiremos cuales de las alternativas propuestas en el modelo de objetivos utilizaremos para la
construcción de este sistema. Una vez completada la elección, se realizaran una serie de documentos
en base a distintas técnicas de especificación, entre las cuales contaran: modelo de casos de uso,
modelo conceptual ampliados con OCL, máquinas de estados finitos (o LTSs/FSMs), y/o diagramas
de actividad. Según consideremos conveniente elegiremos una o otra para representar determinados
detalles de nuestro problema.

A continuación redactaremos las elecciones que tomamos a partir del modelo de objetivos:

\begin{itemize}
\item En el o-refinamiento de la Orden de PM según ranking y cantidad de proyectos actualesla opción Mostrar proveedores de forma inteligente, esto básicamente es la alternativa planteada
en la etapa anterior que permitía al sistema mostrar a los proveedores ordenados por un ranking. El mismo se construye por las encuestas realizadas al final
del proyecto.
\item En el o-refinamiento de Orden de PM según ranking y cantidad de proyectos actuales elegimos PMs ordenados por ranking. Le decisión es similar a la anterior,
utilizamos esto para mostrar los PMs ordenado por ranking generado a partir de las encuestas de fin de proyecto.
\item En el o-refinamiento del seguimiento del proyecto Estado de proyectos actualizados elegimos la opción PM asignado al proyecto
actualiza el estado periódicamente en el sistema.
\item En el o-refinamiento de la finalización del proyecto elegimos la rama que incluye El proyecto se marca como finalizado => se realizan
las encuestas. Esta elección es necesaria para los dos primeros o-refinamientos se puedan tomar.
\end{itemize}

Seleccionadas las alternativas, pasaremos establecer con que modelos representaremos cada detalle de nuestro problema.

 \begin{itemize}
  \item Casos de uso: usaremos esta técnica para representar las interacciones que ocurren entre los distintos
  actores que conforman la empresa(gerente, PM, proveedor, cliente, data entry) y el sistema. Completaremos
  el diagrama con los detalles de los casos de uso, donde presentaremos la serie de pasos que representan una
  determinada interacción con nuestro sistema.
  \item Máquinas de estado: las utilizamos para mostrar como funcionan las actualizaciones de estado que realiza el PM para un proyecto
  durante la etapa de seguimiento (con y sin data entry). Además, mostramos como son las interacciones a través del sistema durante las negociaciones
  del alcance (PM con Cliente) y del presupuesto (Gerente con Cliente).
  \item Diagrama de Actividad:
 \end{itemize}

En las siguiente secciones de este informe detallaremos cada una de estas especificaciones y, además, formularemos cambios respecto a la etapa anterior del
proyecto. 
