El presente Trabajo Práctico se propuso, a partir de los objetivos y requerimientos realizados en el trabajo anterior, 
realizar una serie de modelos que permitan:

\begin{itemize}
	\item Entender en detalle cuales son las interacciones que tendrá el Sistema con el resto de los actores.
	\item Entender cuales son los conceptos que se tienen en cuenta dentro del Sistema, y cuales son las relaciones entre ellos.
	\item Entender cual es el ciclo de vida de un proyecto cuando los diferentes actores utilizan el Sistema.
\end{itemize}

Los cuales creemos que se pudieron realizar de manera adecuada. Encontramos que cada modelo tiene ventajas para ciertos escenarios:
\begin{itemize}
	\item El Diagrama de Casos de Uso (junto con el detalle de cada uno de ellos) es esencial para entender cual es la interfaz del Sistema, y ver cual es el detalle fino de las interacciones definidas en el Diagrama de Contexto.
	\item El Diagrama de Actividad es muy util para entender las interacciones a lo largo del tiempo de varios actores en un proceso determinado. En el caso de este trabajo, se utilizó para mostrar el ciclo de vida completo de un proyecto, y como influye cada actor en las diferentes etapas de este.
	\item El Diagrama de Maquinas de Estado permite dar el detalle de procesos que son muy dinamicos y no son sencillos de expresar con el Diagrama de Actividad. En el caso de este trabajo, se utilizó para mostrar los procesos de: actualizaciones del PM en el seguimiento de un proyecto, y la negociación a través del Sistema del presupuesto (entre Gerente y Cliente) y el alcance del proyecto (entre PM y Cliente).
	\item El Modelo Conceptual muestra cuales son los conceptos a tener en cuenta en un problema, y como se relacionan entre ellos. En el caso de este trabajo, se utilizó para mostrar las relaciones que tienen los distintos actores alrededor del concepto ``Proyecto'', ya que es un concepto central del problema y merece ese nivel de detalle.
\end{itemize}

En cuanto a la trazabilidad entre los distintos diagramas planteados, vimos que muchas veces es dificil mantener la coherencia entre estos, pero que es fundamental para lograr una comprensión entera del problema. Incluso en el caso en el que algunos diagramas definian escenarios similares y se superponían, utilizar diferentes diagramas nos permitió ver estos escenarios desde distintos niveles de perspectiva.

La experencia que se lleva el grupo de las distintas herramientas utilizadas es que cada una debe ser usada en el contexto adecuado, ya que la realización de los diagramas lleva un tiempo considerable que no parecería justificarse si el escenario no es lo suficientemente complicado, o si el diagrama no aporta mucho más que leer el problema en lenguaje natural.