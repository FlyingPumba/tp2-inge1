Nuestra metodología, a la hora de pensar y hacer el presente trabajo práctico, fue, a partir del primer proceso de elicitación, comenzar por un listado de los procesos actuales de la empresa, puntualizando cuales de ellos eran los que traían problemas de escalabilidad. Luego confeccionamos el Diagrama de Contexto, el cual nos permitió lograr una primera aproximación a los requerimientos del sistema a proponer, mediante la visualización de las interacciones entre los diversos agentes. Después comenzamos con el Diagrama de Objetivos, confeccionándolo mediante la metodología top-down, utilizando como guía los elementos ya descriptos. Tanto el Diagrama de Contexto como el de Objetivos requirieron varias iteraciones para su finalización. Vale la pena mencionar que, luego del segundo proceso de elicitación, debimos introducir varios cambios sobre nuestro sistema, ya que nuestra idea original no contemplaba diversos requerimientos exigidos por el cliente, requerimientos que no habían quedado claros en el primer proceso de elicitación. Otro hecho destacado fue la confección de diversos casos de uso, los cuales nos sirvió como guía a la hora de homogeneizar la información contenida en ambos Diagramas.

Con respecto a las dificultades encontradas haciendo el presente trabajo práctico, consideramos que la confección del Diagrama de Objetivos fue la principal dificultad. No solo por la dificultad de pensar su contenido, es decir pensar los objetivos y sus subsiguientes refinamientos, sino también su elaboración gráfica, ya que las diversas iteraciones implicaron cambios que volvieron engorroso la visualización del mismo debido al tamaño que fue adquiriendo el Diagrama. Estos factores nos hacen dudar de su utilidad para la práctica profesional, partiendo del hecho de que el trade-off entre trabajo que implica confeccionarlo y su utilidad no parece ser positivo. Sin embargo no descartamos que esto se deba a nuestro propia inoperancia, dado que esta fue nuestra primera experiencia con este tipo de Diagrama.
