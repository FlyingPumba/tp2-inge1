Cambios con respecto al TP1


En base al feedback obtenido de la empresa \textit{DC Construcciones}, y un analísis detenido de los requerimientos, llegamos a la conclusión de que era necesario realizar los siguientes cambios a los objetivos planteados en la primera etapa:

\begin{itemize}
  \item El procedimiento por el cual el sistema logra saber que un proyecto a comenzado, necesario para poder exigir, posteriormente, actualizaciones al PM, no estaba incluido en las asunciones. Establecemos que el PM es quien marca, en el sistema, como iniciado el proyecto.
  \item Para el caso en el que un proveedor cancela durante la realización de un proyecto, se había tomado la asunción de que el proveedor cancela a través del sistema, pero esto va en contra de las prácticas comerciales de sentido comun. Por esta razón, ahora asumimos que el proveedor cancela directamente al PM, quien después refleja esto en el Sistema.
  \item Una vez que el PM informa que canceló un proveedor en el Sistema, este envía una notificicación al Gerente. Dado que puede haber alguna parte del proyecto ya realizada, el PM redefine el alcance antes de elegir un nuevo proveedor, y luego evalua negativamente en el Sistema al proveedor que canceló.
  \item El cambio del PM de un proyecto a partir de la manifestación del descontento del cliente no estaba incluido en nuestra primera entrega. En este trabajo práctico estipulamos que este tipo de cambio se origina con la manifestación del descontento del cliente directamente al Gerente. Luego el gerente se comunica directamente con el PM desplazado comunicandole su desplazamiento del proyecto. Acto seguido el gerente evalua al PM desplazado en el sistema e inicia un nuevo proceso de selección de PM. Este proceso es identico al descripto en la primera etapa, con la unica diferencia de que el sistema notifica al cliente una vez que fue seleccionado otro PM. Es importante mencionar que el proveedor sigue trabajando durante todo este proceso, ya que el intervalo entre el desplazamiento y la seleccion de otro PM es muy corto.
  \item Relacionado con el item anterior, consideramos que la situación donde el PM cancela por motus propio no sucede, por lo cual no es necesario especificarla.  
En base al feedback obtenido de la empresa \textit{DC Construcciones}, y un análisis detenido de los requerimientos, llegamos a la conclusión de que era necesario realizar los siguientes cambios a los objetivos planteados en la primera etapa:

CONFLICTO DESPUES LO ARREGLO (JOSÈ)
\begin{itemize}
  \item Para el caso en el que un proveedor cancela durante la realización de un proyecto, se había tomado la asunción de que el proveedor cancela a través del sistema, pero esto va en contra de las prácticas comerciales de sentido común. Por esta razón, ahora asumimos que el proveedor cancela hablandole en persona al PM, quien después refleja esto en el Sistema.
  \item Una vez que el PM informa que canceló un proveedor en el Sistema, este envía una notificicación al Gerente. Dado que puede haber alguna parte del proyecto ya realizada, el PM redefine el alcance antes de elegir un nuevo proveedor, y luego evalúa negativamente en el Sistema al proveedor que canceló.
  \item El gerente debe validar nuevamente el proyecto ante el cambio de proveedor y redefinición del alcance.
  \item El Cliente puede comunicar un malestar con el PM, en cuyo caso el PM será apartado del proyecto (esto se realiza en forma personal). El PM también puede apartarce por voluntad propia, y se hace de forma personal.
  \item El sistema notifica al Cliente del cambio de PM.
  \item Cuando el PM es apartado de un proyecto, el gerente se lo evalúa negativamente.

\end{itemize}

Estos cambios serán reflejados en nuestras especificaciones.
