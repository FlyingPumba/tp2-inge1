En base al feedback obtenido de la empresa \textit{DC Construcciones}, y un análisis detenido de los requerimientos, llegamos a la conclusión de que era necesario realizar los siguientes cambios a los objetivos planteados en la primera etapa:

\begin{itemize}
  \item El procedimiento por el cual el sistema logra saber que un proyecto a comenzado, necesario para poder exigir, posteriormente, actualizaciones al PM, no estaba incluido en las asunciones. Establecemos que el PM es quien marca, en el sistema, como iniciado el proyecto.
  \item Para el caso en el que un proveedor cancela durante la realización de un proyecto, se había tomado la asunción de que el proveedor cancela a través del sistema, pero esto va en contra de las prácticas comerciales de sentido común. Por esta razón, ahora asumimos que el proveedor cancela comunicandose directamente con el PM, quien después refleja esto en el Sistema.
  \item Una vez que el PM informa que canceló un proveedor en el Sistema, este envía una notificación al Gerente. Dado que puede haber alguna parte del proyecto ya realizada, el PM redefine el alcance antes de elegir un nuevo proveedor. Previo a redifinir el alcance el PM evalúa negativamente en el Sistema al proveedor que canceló.
  \item El gerente debe validar nuevamente el proyecto ante el cambio de proveedor y la redefinición del alcance.
  \item El Cliente puede comunicar un malestar con el PM, en cuyo caso el PM será apartado del proyecto (este desplazamiento se realiza por fuera del sistema). El PM también puede apartarce por voluntad propia, accion que realiza comunicandose directamente con el gerente.
  \item El sistema notifica al Cliente del cambio de PM.
  \item Cuando el PM es apartado de un proyecto, el gerente lo evalúa negativamente.

\end{itemize}

Estos cambios serán reflejados en nuestras especificaciones.
